\documentclass[a4paper, titlepage]{article}

\usepackage{courier} % Required for the courier font
\usepackage{listings}
\usepackage{graphicx}
\usepackage[utf8]{inputenc}
\usepackage[bookmarks]{hyperref}

\lstset{
mathescape=true,
basicstyle=\ttfamily,
numbers=left
}

\begin{document}

\title{Exercise Lecture 4}
\author{Sigurt Dinesen \\sidi@itu.dk \\\\ Hans Pagh \\hkkp@itu.dk 
\\\\Jens Egholm \\jegp@itu.dk}
\maketitle
\pagebreak

\section*{Evaluation}
\subsection*{Minimax}
The core algorithm used is minimax, with $\alpha/\beta$ pruning and caching. To
comply with the assignment constraint of spending no more than 10 seconds to
decide a move, we use iterative deepening with an incrementing depth and a timer
to cut off the search. Since the search can potentially be cut off before
traversing the entire tree, the iterative deepening ensures an evenly
distributed heuristic in contrast to a depth-first approach.

\subsubsection*{Cache}
We have implemented a cache to store states that have been explored previously.
Adding a state to the cache also adds it's mirrored state to the cache,
decreasing the number of visited nodes further.

\subsubsection*{Stalling and Quick win}
When our AI reaches the bottom of our search, it can sure tell if its going to
lose or win. But this is only against a optimal player. So even though the AI
knows it has lost, it makes moves that will make the game as long as possible to
make room for the suboptimal opponents to make mistakes. The same
mechanism\footnote{The implementation uses a trick in the \texttt{utility}
method, which both affects the h-values of winning and losing states.} also
allows for the heuristic to move towards winning states, just to spare the
opponent of any (further) humiliation.

\section*{Heuristics}
\subsection*{Knowledge base}
Because the depth of the state space is to large to efficiently exhaust,
heuristics are needed to guide the search. Still, it proved difficult to make a
heuristic to guide the first couple moves, as the sparsely filled states leave
little to go on. For this reason we opted for a knowledge base (KB) to guide the
first few moves towards optimal paths. Concretely, we use a KB provided by John
Tromp\footnote{http://homepages.cwi.nl/\textasciitilde tromp/}. The KB consists
of ply-8 moves which, in itself, is insufficient for expert play. The initial
guidance, leading to optimal states up to ply-8, makes it easier to find good
hueristics, as it is a lot easier to make choices with e.g. 21 coins placed than
with 5. There is still a few moves after the KB where our hueristic potentially
makes bad choice. We cosidered using a Moves To Win heuristic in the begginning,
but it proved to be no better than the Threats heuristic.


\subsection*{Threats}
The Threats heuristic evaluates the state of a board based on the number of
threats on the board. A threat is defined as position where either player can
win the game, which simply means a position where either plays have 3 connected
coins and an open position to put the last coin. The threats are then further
categorized into even and odd threats, based on the row they where found in. By
analysing these threats we can force the opponent to play moves in the late
game, that makes the AI win. An example of this is when there is only one column
left, the opponent is forced to play in this column, earlier in the game the AI
made sure to make a threat in this column so it will win in this scenario. By
investigating strategies and tactics online we found the following rules to
evaluate the threats on the board.
\footnote{http://asingleneuron.com/2012/08/04/connect-four/}

In the following player A is the player who plays the first coin
\begin{itemize} 
	\item If player A had an odd threat, player B had no even threats below this in the same column and player B had no odd threats anywhere else, it was a win for A
	\item If player A had a greater number of odd threats than player B had odd threats and player B had no even threats, it was a win for A	
	\item If neither of these were true and player B had any even threats it was win for B
	\item Else it was a draw state.
\end{itemize} 

\subsection*{Moves to win}
The moves-to-win (MTW) heuristic is based on calculating the number of moves required to 
secure a win. Every time a move is made the heuristic calculates all the possible 
combinations to win a game and returns a weight, which is higher (better) the fewer
moves are required to win. 

It uses a data structure which is updated whenever a move is made by first 
removing any MTW combinations from the opponent which included the new move (that MTW is 
now no longer available). Second the algorithm finds and stores the combinations where 
there space enough for a possible win.
This way of caching gives a time advantage because the heuristic only has to calculate new MTW 
combinations for one coin at the time, and not the entire board after each move.

These MTW values are updated for both players, but a sligthly larger weight is put on the 
MTW value of the opponent to make sure that an immediate win from the \texttt{max} player 
is given the same weight as an immediate win from the \texttt{min} player. Since the values 
are calculated immediately after the \texttt{max} player made her move, the next move cannot
be made before \texttt{min} moves, and so \texttt{max} would lose.

The heuristic is strong in the beginning because it tries to get as long series of coins
as possible, while preventing the opponent to grow large MTW combinations. It is 
relatively weak in the end-game because it fails to detect some of the late-game challenges 
described in the \textit{Threats} heuristic. 

There are a number of ways this algorithm can be improved. First of all it could be
interesting to add some additional value to MTW combinations that is close to the bottom of
the board or combinations where there are coins below the combination coordinates 
(because they require fewer plys to reach). It could also be interesting to take advantage
of multithreading by splitting the calculations into different subtrees and hand them to 
parallel processes. This would give a better runtimes and, in turn,
cover a larger search space\footnote{This also applies to the \texttt{Threats} algorithm
of course.}. There are other means of optimization and, in sum, the heuristics have a 
number of possible improvements that could be interesting to explore in the future.

\end{document}

