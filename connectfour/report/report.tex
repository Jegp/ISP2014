\documentclass[a4paper, titlepage]{article}

\usepackage{courier} % Required for the courier font
\usepackage{listings}
\usepackage{graphicx}
\usepackage[utf8]{inputenc}
\usepackage[bookmarks]{hyperref}

\lstset{
mathescape=true,
basicstyle=\ttfamily,
numbers=left
}

\begin{document}

\title{Exercise Lecture 4}
\author{Sigurt 'Slap Blære' Dinesen \\sidi@itu.dk \\\\ Hans Pagh \\hkkp@itu.dk 
\\\\Jens Egholm \\jegp@itu.dk}
\maketitle
\pagebreak

\section*{Evaluation}
\subsection*{Minimax}
For evaluating the heuristics we use the minimax algorithm with alpha-beta pruning to
search for the heuristics with maximum utility and minimum utility recursively. 
For searching we use iterative deepening with an incrementing depth and a timer to cut off
the search. Since the search can potentially be cut off before traversing the entire tree, 
the iterative deepening ensures an evenly distributed heuristic in contrast to a depth-first
approach.

\section*{Heuristics}
\subsection*{Knowledge base}
Because the depth of the state space is to large to efficiently exhaust,
heuristics are needed to guide the search. Still, it proved difficult to make a
heuristic to guide the first couple moves, as the sparsely filled states leave
little to go on. For this reason we opted for a knowledge base (KB) to guide the
first few moves towards optimal paths. Concretely, we use a KB provided by John
Tromp\footnote{http://homepages.cwi.nl/\textasciitilde tromp/}. The KB consists
of ply-8 moves which, in itself, is insufficient for expert play. The initial
guidance, leading to optimal states up to ply-8, makes it easier to find good
hueristics, as it is a lot easier to make choices with e.g. 21 coins placed than
with 5. There is still a few moves after the KB where our hueristic potentially
makes bad choice. We cosidered using a Moves To Win heuristic in the begginning,
but it proved to be no better than the Threats heuristic.


\subsection*{Threats}
The Threats heuristic evaluates the state of a board based on the number of threats on the board. 
A threat is defined as position where ether player can win the game, which simply means a position
where ether plays have 3 connected coins and an open position to put the last coin. 
The threats are then further categories into even and odd threats, based on that row they where found in.
 
All the threats are then evaluated based on the following rules
\begin{itemize} 
	\item 
	\item
	\item
\end{itemize} 
By following these rules the AI, will try to force the opponent into playing a move, which will make the AI win. 

\subsection*{Moves to win}
The moves-to-win (MTW) heuristic is based on calculating the number of moves required to 
secure a win. Every time a move is made the heuristic calculates all the possible 
combinations to win a game. First it removes any MTW combinations from the opponent
the new coin was a part of (that MTW is now no longer available) and second it finds 
the combination where there is \texttt{a)} space enough for a possible win and 
\texttt{b)} fewest coins required to win in comparison to others.

These MTW values are updated for both players, but a sligthly larger weight is put on the 
MTW value of the opponent to make sure that an immediate win from the \texttt{max} player 
is given the same weight as an immediate win from the \texttt{min} player. Since the values 
are calculated immediately after the \texttt{max} players turn, the next move cannot be 
made before \texttt{min} moves, and so \texttt{max} would lose.

The heuristic uses a datastructure that is passed down recursively via the 
\texttt{minimax} algorithm, so the heuristic can cache the MTW combinations for each player.
This gives a time advantage because the heuristic only has to calculate new MTW 
combinations for one player at the time.

The heuristic is strong in the beginning because it tries to get as long series of coins
as possible, while preventing the opponent to grow large MTW combinations. It is 
relatively weak in the end-game because it fails to detect some of the challenges 
described in the \textit{Threats} heuristic.

\end{document}

