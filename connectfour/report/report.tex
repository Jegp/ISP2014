\documentclass[a4paper, titlepage]{article}

\usepackage{courier} % Required for the courier font
\usepackage{listings}
\usepackage{graphicx}
\usepackage[utf8]{inputenc}
\usepackage[bookmarks]{hyperref}

\lstset{
mathescape=true,
basicstyle=\ttfamily,
numbers=left
}

\begin{document}

\title{Exercise Lecture 4}
\author{Sigurt Dinesen \\sidi@itu.dk \\\\ Hans Pagh \\hkkp@itu.dk 
\\\\Jens Egholm \\jegp@itu.dk}
\maketitle
\pagebreak

\section*{Evaluation}
\subsection*{Minimax}
For evaluating the heuristics we use the minimax algorithm with alpha-beta pruning to
search for the heuristics with maximum utility and minimum utility recursively. 
For searching we use iterative deepening with an incrementing depth and a timer to cut off
the search. Since the search can potentially be cut off before traversing the entire tree, 
the iterative deepening ensures an evenly distributed heuristic in contrast to a depth-first
approach.
\subsubsection{Cache}

\subsubsection{Stalling and Quick win}

\section*{Heuristics}
\subsection*{Knowledge base}	
This is bullshit. By Sigurt. And Hans. Not Jens.

\subsection*{Threats}
The Threats heuristic evaluates the state of a board based on the number of threats on the board. 
A threat is defined as position where ether player can win the game, which simply means a position
where ether plays have 3 connected coins and an open position to put the last coin. 
The threats are then further categories into even and odd threats, based on that row they where found in.
By analysing these threats we can force the opponent to play moves in the late game, that makes the AI win.
An example of this is when there is only one column left, the opponent is forced to play in this column, 
earlier in the game the AI made to sure to make a threat in this column so it will win in this scenario. 
By investigating strategies and tactics online we found the following rules to evaluate the threats on the board.
In the following player A is the player how plays the first coin
\begin{itemize} 
	\item If player A had a on odd threat and player B had no even threats below this in the same column and player B had no odd threats anywhere else, it was a win for A
	\item If player A had a greater number of odd threats than player B had odd threats and player B had no even threats, it was a win for A	
	\item If nether of these was true and player B had any even threats it was win for B
	\item Else is was a draw state.
\end{itemize} 

\subsection*{Moves to win}
The moves-to-win (MTW) heuristic is based on calculating the number of moves required to 
secure a win. Every time a move is made the heuristic calculates all the possible 
combinations to win a game. First it removes any MTW combinations from the opponent
the new coin was a part of (that MTW is now no longer available) and second it finds 
the combination where there is \texttt{a)} space enough for a possible win and 
\texttt{b)} fewest coins required to win in comparison to others.

These MTW values are updated for both players, but a sligthly larger weight is put on the 
MTW value of the opponent to make sure that an immediate win from the \texttt{max} player 
is given the same weight as an immediate win from the \texttt{min} player. Since the values 
are calculated immediately after the \texttt{max} players turn, the next move cannot be 
made before \texttt{min} moves, and so \texttt{max} would lose.

The heuristic uses a datastructure that is passed down recursively via the 
\texttt{minimax} algorithm, so the heuristic can cache the MTW combinations for each player.
This gives a time advantage because the heuristic only has to calculate new MTW 
combinations for one player at the time.

The heuristic is strong in the beginning because it tries to get as long series of coins
as possible, while preventing the opponent to grow large MTW combinations. It is 
relatively weak in the end-game because it fails to detect some of the challenges 
described in the \textit{Threats} heuristic.

\end{document}

