\documentclass[12pt,a4paper]{article}
\usepackage[mathletters]{ucs}
\usepackage[utf8]{inputenc}
\usepackage{amsmath}
\usepackage{amsfonts}
\usepackage{amssymb}
\usepackage{lmodern}
\usepackage{listings}
\DeclareUnicodeCharacter{22AD}{\modelsn}

\author{Jens Egholm Pedersen}
\begin{document}

\section*{Problem 1}
\subsection*{a)}
We set $f = g(n)$ such that $g$ is the minimum remaining driving length.
Except for the first state I have excluded nodes where curved pieces
are flipped to the other side, because the flipped pieces are 'travelling'
in the wrong direction, such that the minimum remaining distance cannot possibly
be less than non-flipped pieces.

See attachment 1.A.

\subsection*{b)}
As above I have avoided drawing flipped pieces, except in the first state.

See attachment 1.B.

\subsection*{c)}
The SLD heuristic is admissible because it fulfills the criteria of never
overestimating the cost to reach the goal. The shortest distance between
two points will always be the straight-line distance, so the found path must
be the best solution in a 2-dimensional space.

\subsection*{d)}
Consistency can be defined via the triangle inequality, which shows that
each side of a triangle cannot be longer than the sum of the other two sides.
This holds in our case because of the optimistic heuristic and the restrictions
placed on us by the 2-dimensional rail-grid.
The SLD is consistent because there is no way to make a shorter route than the
one already found by A*.

\section*{Problem 2}
\subsection*{a)}
See attachment 2.A. The \texttt{UtilitySum} decison would be to set an
\texttt{X} in the lower left corner, because the utility value is 1
compared to the other two states which both sum to the utility value 0.

\subsection*{b)}
\begin{lstlisting}[frame=single]
function UtilitySum(s):
    if isEndNode(s) then
        return UtilityValue(s)
    else
        children <- getChildren(s)
        sum <- 0
        for each node in children do
            sum <- sum + UtilitySum(s)
        return sum
\end{lstlisting}
This function takes a state as a parameters and then examines if it is an
end-node. If so, we simply return the utility value from that state, which
will be within the range [-1; 1].
If it is not an end-node, we simply cumulate the utility values from
all the child-nodes recursively, which will give us the sum of all the
utility values in a depth-first manor.

\subsection*{c)}
The critical mistake is that \texttt{UtilitySum} does not take into
account that there are different players with different goals. In
other words the cumulated utility value has different meaning depending
on the player. In attachment 2.C a (simplified and somewhat hypothetical)
game tree shows a
situation where this distinction matters. State \texttt{b)} and \texttt{c)}
have the summed value of 0, so the state should not be favoured. But in fact
both \texttt{MAX} and \texttt{MIN} can win if they enter this state: if
\texttt{MAX} starts she wins and vice versa. This can also be illustrated
by pinning a \texttt{UtilitySum} player against a \texttt{Minimax} player.
\texttt{UtilitySum} will not have any reservations to enter state \texttt{a)},
while \texttt{Minimax} immediately will see that this is either a winning
or losing move.

Since \texttt{UtilitySum} is not
able to account for the importance of whos turn it is, it cannot play as
optimal as \texttt{Minimax}.

\section*{Problem 3}
\subsection*{a)}
See attachment 3.A.

\subsection*{b)}
Slack form:
\begin{align*}
x_3 & = 4 - 1/2x_1 - x_2 \\
x_4 & = 4 - x_1 \\
x_5 & = -2 + 2x_1 + x_2 \\
z & = 2x_1 + x_2 \\
x_1, x_2, x_3, x_4, x_5 & \ge 0
\end{align*}

\subsection*{c)}
The initial dictionary is only feasible if the constraints can be
fulfilled by setting $x_1 = 0, x_2 = 0, ..., x_n = 0$. Since $x_5$ has
a negative constant, the slack form is not feasible.

\subsection*{d)}
For the two-phase simplex we maximize
$w = -x_0$
in the following:
\begin{align*}
x_3 & = 4 - 1/2x_1 - x_2 + x_0\\
x_4 & = 4 - x_1 + x_0 \\
x_5 & = -2 + 2x_1 + x_2  + x_0\\
x_0, x_1, x_2, x_3, x_4, x_5 & \ge 0
\end{align*}

Pivoting $x_5$:
\begin{align*}
x_5 & = -2 + 2x_1 + x_2  + x_0 \Leftrightarrow \\
-x_0 & = -2 + 2x_1 + x_2 - x_5 \Leftrightarrow \\
x_0 & = 2 - 2x_1 -x_2 + x_5
\end{align*}

Inserting $x_0$:
\begin{align*}
x_3 & = 4 - 1/2x_1 - x_2 + 2 - 2x_1 -x_2 + x_5 \\
    & = 6 - 5/2x_1 - 2x_2 + x_5 \\
x_4 & = 4 - x_1 + 2 - 2x_1 -x_2 + x_5 \\
    & = 6 - 3x_1 - x_2 + x_5 \\
\end{align*}

Since there are no negative constants, the dictionary is feasible.
We now choose $x_1$ as the entering variable and $x_0$ as the leaving,
because $x_1$ has the most negative coefficient:
\begin{align*}
x_0 &= 2 - 2x_1 - x_2 + x_5 \Leftrightarrow \\
2x_1 &= 2 - x_2 + x_5 - x_0 \Leftrightarrow \\
x_1  &= 1 - 1/2x_2 + 1/2x_5 - 1/2x_0
\end{align*}

\begin{align*}
x_3 & = 6 - 5/2(1 - 1/2x_2 + 1/2x_5 - 1/2x_0) - 2x_2 + x_5 \\
    & = 6 - 5/2 + 5/4x_2 - 5/4x_5 - 5/4x_0 - 2x_2 + x_5 \\
    & = 7/2 - 3/4x_2 + 1/4x_5 - 5/4x_0 \\
\\
x_4 & = 6 - 3(1 - 1/2x_2 + 1/2x_5 - 1/2x_0) - x_2 + x_5 \\
    & = 6 - 3 + 3/2x_2 - 3/2x_5 + 3/2x_0 - x_2 + x_5 \\
    & = 3 + 1/2x_2 - 1/2x_5 + 3/2x_0 \\
\\
z & = 2x_1 + x_2 \\
  & = 2(1 - 1/2x_2 + 1/2x_5 - 1/2x_0) + x_2 \\
  & = 2 - x_2 + x_5 - x_0 + x_2 \\
  & = 2 + x_5 \\
\end{align*}

We then have the dictionary:
\begin{align*}
x_1 & = 1 - 1/2x_2 + 1/2x_5 \\
x_3 & = 7/2 - 3/4x_2 + 1/4x_5 \\
x_4 & = 3 + 1/2x_2 - 1/2x_5 \\
z   & = 2 + x_5
\end{align*}
Which is the same as the dictionary given in problem \texttt{d)}. 

\subsection*{e)}
The initial dictionary shows that $x_2$ is zero (since it is not present as
a slack variable) and that $x_1$ is set to the value of the constant: 1. The
corner we start at is then given by the coordinates: $(1, 0)$. This also
aligns with the expectation that $x_0$ should be minimised as much as possible.

\subsection*{f)}
By coordinates ($x_1, x_2$) the $z'$ values are:
\begin{align*}
(1, 0) &: z' = 1^2 + 0^2 = 1 \\
(4, 0) &: z' = 4^2 + 0^2 = 16 \\
(4, 2) &: z' = 4^2 + 2^2 = 20 \\
(0, 2) &: z' = 0^2 + 2^2 = 4 \\
(0, 4) &: z' = 0^2 + 4^2 = 16 \\
\end{align*}

\subsection*{g)}
The neighbors to the corner point $(1, 0)$ are $(4, 0)$ and $(0, 2)$ because
the points can be reached by maximising only one constraint.

\subsection*{h)}
To reach each corner point, the variable that describes the line towards that
corner, must be pivoted. As an example we can assume we are at the coordinates
$(4, 0)$ in attachment 3.A. To reach $(4, 2)$ we must maximise $x_4$ because
that is the variable that is currently constraining growth. By isolating $x_4$
and pivoting it in, we are able to describe a system where we can maximise $x_4$
until another variable starts constraining (unless the problem is unbounded).

\subsection*{i)}
The hill-climbing algorithm will continue to 'climb' up the neighbors and pick
the best solution available. Since we are optimising $z' = x_1^2 + x_2^2$
the corner $(4, 0)$ (value of 16) will be preferable to the corner $(2, 0)$
(value of 4).

\subsection*{j)}
Written in order after the two-phase algorithm (beginning in $(1, 0)$):
\begin{tabular}{l l}
$(2, 0)$ & Discarded\\
$(4, 0)$ & Chosen \\
$(4, 2)$ & Chosen \\
$(0, 4)$ & Discarded \\
         & $(4, 2)$ wins
\end{tabular}
After visiting $(4, 0)$ the algorithm cannot find any better neighbors and
terminates.

I have assumed that the hill-climbing algorithm is smart enough not to visit
previous (inferior) points.

\end{document}
