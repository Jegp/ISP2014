\documentclass[12pt,a4paper]{article}
\usepackage[mathletters]{ucs}
\usepackage[utf8]{inputenc}
\usepackage{amsmath}
\usepackage{amsfonts}
\usepackage{amssymb}
\usepackage{lmodern}
\usepackage{listings}
\DeclareUnicodeCharacter{22AD}{\modelsn}

\author{Jens Egholm Pedersen}
\begin{document}

\section*{Problem 1}


\section*{Problem 1}

\subsection*{a)}
See attachment 2.A. The \texttt{UtilitySum} decison would be to set an
\texttt{X} in the lower left corner, because the utility value is 1
compared to the other two states which both sum to the utility value 0.

\subsection*{b)}
\begin{lstlisting}[frame=single]
function UtilitySum(s):
    if isEndNode(s) then
        return UtilityValue(s)
    else
        children <- getChildren(s)
        sum <- 0
        for each node in children do
            sum <- sum + UtilitySum(s)
        return sum
\end{lstlisting}
This function takes a state as a parameters and then examines if it is an
end-node. If so, we simply return the utility value from that state, which
will be within the range [-1; 1].
If it is not an end-node, we simply cumulate the utility values from
all the child-nodes recursively, which will give us the sum of all the
utility values in a depth-first manor.

\subsection*{c)}
The critical mistake is that \texttt{UtilitySum} does not take into
account that there are different players with different goals. In
other words the cumulated utility value has different meaning depending
on the player. In attachment 2.C a (simplified and somewhat hypothetical)
game tree shows a
situation where this distinction matters. State \texttt{b)} and \texttt{c)}
have the summed value of 0, so the state should not be favoured. But in fact
both \texttt{MAX} and \texttt{MIN} can win if they enter this state: if
\texttt{MAX} starts she wins and vice versa. This can also be illustrated
by pinning a \texttt{UtilitySum} player against a \texttt{Minimax} player.
\texttt{UtilitySum} will not have any reservations to enter state \texttt{a)},
while \texttt{Minimax} immediately will see that this is either a winning
or losing move.

Since \texttt{UtilitySum} is not
able to account for the importance of whos turn it is, it cannot play as
optimal as \texttt{Minimax}.

\section*{Problem 3}

\end{document}
